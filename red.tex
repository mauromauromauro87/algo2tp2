%!TEX root = tp2.tex
\section{Red}
El módulo red permite crear una red de computadoras, agregar nuevas, conectarlas y averiguar el camino mínimo entre dos de ellas.


\subsection{Interfaz}

\sexc{Red}
\generos{red}

\subsubsection*{Operaciones}

\textit{Aclaración: n=$\# (r.compus)$, L=Longitud de IP más larga}

\operacion{Nueva}{}{res : red}
 {true}
 {$\var{res} \equiv$ iniciarRed}
 {Crea una red vacía}
 {O(1)}

\operacion{InterfazUsada}{in r:red, in c:compu, in c1:compu}{res:interfaz}
{$\neg(c = c1) \land c \in compus(r) \land c1 \in compus(r)$}
{$res \igobs InterfazUsada(r,c,c1)$}
{Retorna por copia la interfaz de la primer compu recibida por parámetro usada para conectarse con la segunda}
{O($\#$(compus(r)))}

\operacion{Compus}{in r:red}{res:conj(compu)}
{$true$}
{$res \igobs computadoras(r)$}
{Devuelve el conjunto de compus}
{O(1)}
{Retorna el conjunto de computadoras por referencia}

\operacion{Conectadas?}{in r:red, in c:compu, in c1:compu}{res:bool}
{$\neg(c = c1) \land c \in compus(r) \land c1 \in compus(r)$}
{$res \igobs conectadas?(r)$}
{Indica si dos compus estan conectadas}
{O($\#$compus(r))}
{}

\operacion{AgregarCompu}{in/out r : red,in c : compu}{}
 {$r \equiv r_0$ $\land$ ($\forall$ $\var{$c_{1}$}$:compu)($\var{$c_{1}$}$ $\in$ computadoras($\var{r}$))$\implies$ IP($\var{c}$) $\neq$ IP($\var{$c_{1}$}$)}
 {$\var{r} \equiv$ agregarComputadora($\var{r},\var{c}$)}
 {Agrega a la red $\var r$ la computadora $\var c$}
 {O(n + copy(c))}

\operacion{Conectar}{in/out r : red,in c1 : compu,in i1 : interfaz,in c2 : compu,in i2 : interfaz}{}
 {$r \equiv r_0$ $\land$ $\var{$c_{1}$}$ $\in$ computadoras($\var{r}$) $\land$ $\var{$c_{2}$}$ $\in$ computadoras($\var{r}$) $\land$ IP(c1) $\neq$ IP(c2) \\ $\land$ $\neg$conectadas(r,c1,c2) $\land$ $\neg$UsaInterfaz?(r,c1,i1) $\land$ $\neg$UsaInterfaz?(r,c2,i2)}
 {$\var{r} \equiv$ conectar($\var{r},\var{c1},\var{i1},\var{c2},\var{i2}$)}
 {Conecta las computadoras $\var{$c_{1}$}$ y $\var{$c_{2}$}$}
 {O(n)}

\operacion{CaminoMinimo}{in r : red, in f : compu, in d : compu}{res : secu(compu)}
 {$\var{$c_{1}$}$ $\in$ computadoras($\var{r}$) $\land$ $\var{$c_{2}$}$ $\in$ computadoras($\var{r}$)}
 {$\var{r} \equiv$ caminosMinimos($\var{r}$,$\var{$c_{1}$}$,$\var{$c_{2}$}$)}
 {Devuelve los caminos mínimos entre dos computadoras}
 {O($L*n^3$)}

\operacion{StringAIndice}{in r : red,in c : compu}{res : nat}
 {$\var{c}$ $\in$ computadoras($\var{r}$)}
 {$\var{res}$ $\equiv$ indice($\var{r}$,$\var{c}$)}
 {Devuelve un nat distinto para cada IP, una vez creada la red siempre retorna el mismo para cada compu}
 {O(n)}

\operacion{UsaInterfaz?}{in r : red,in c : compu,in i : interfaz}{res : bool}
 {$\var{c}$ $\in$ computadoras($\var{r}$) $\land$ $\var{i}$ $\in$ interfaces($\var{c}$)}
 {$\var{res} \equiv UsaInterfaz?(r,c,i)$}
 {Devuelve TRUE si y solo si la interfaz está siendo utilizada}
 {O(n)}

\operacion{HayCamino?}{in r : red,in c1 : compu,in c2 : compu}{res : bool}
 {$\var{$c_{1}$}$ $\in$ computadoras($\var{r}$) $\land$ $\var{$c_{2}$}$ $\in$ computadoras($\var{r}$)}
 {$\var{res} \equiv \neg(Vacia?(caminoMinimo(r,c1,c2)))$}
 {Devuelve TRUE sí y solo sí hay un camino válido entre las dos computadoras}
 {O($L*n^3$)}

\subsection{Representación}
\serc{redstr}{
		\donde{redstr}{\tupla{
				Compus : \mbox{conj(compu)},
				Conexiones : \mbox{dicc(compu,dicc(interfaz,compu))}
		}}
		\donde{compu}{\tupla{
			IP : \mbox{string},
			interfaces : \mbox{conj(interfaz)}
		}}
		\donde{interfaz}{nat}
}

\subsection{Invariante de representación}
\begin{enumerate}
\item El conjunto de claves de conexiones está incluído o es igual al conjunto de compus
\item 'Simetría en las conexiones' : Cada una de las claves de conexiones está en los significados de las compus a las que está conectada
\item Las compus conectadas a c están en las claves de las conexiones
\item Las claves del significado de cada una de las compus 'c' está incluída en el conjunto de interfaces de 'c'
\end{enumerate}

\subsubsection*{Rep.}
\begin{enumerate}
\item[] $claves(conexiones(r)) \subseteq compus(r)$ $\land$
\item[] $((\forall c : compu)$ $c \in claves(r.conexiones)) \impluego$ \\
		\indent $\ \ claves(significado(r.conexiones,c))$ $\subseteq$ $c.interfaces$ $\land$ \\
		\indent $\ \ ((\forall c_1 : compu) c_1 \in significados(conexiones(r),c))$ $\impluego$ \\
		\indent $\ \ c_1 \in claves(r.conexiones)$ $\yluego$ $c \in significados(conexiones(r),c_1)$ \\
\end{enumerate}

\subsection{Función de abstracción}

\abs{redstr}{red}{rstr}{r} \\
$computadoras(r) \igobs rstr.compus$ $\land$ \\
$(\forall c_0,c_1:compu)$ $conectadas?(r,c_0,c_1)$ $\igobs$ $c_1 \in significados(rstr.conexiones,c_0)$ $\land$\\
$interfazUsada(r,c_0,c_1)$ $\igobs$ $dameClave(significado(rstr.conexiones,c_0),c_1)$

\subsubsection*{Extensión TAD diccionario}
\begin{description} 
\item[] $significados:$ $dicc(compu,dicc(interfaz,compu))$ $d$ $\times$ $compu$ $c$ $\rarr$ $conj(compu)$ \setlength{\parindent}{1cm} \indent
\item[] $significados(d)$ $=$  $obtenerSignificados(significado(d,c),claves(significado(d,c)))$
\\
\item[] $obtenerSignificados:$ $dicc(interfaz,compu)$ $d$ $\times$ $conj(interfaz)$ $ci$ $\rarr$ $conj(compu)$ \setlength{\parindent}{0.1cm} \indent
\item[] $obtenerSignificados(d,c)$ $=$
	\setlength{\parindent}{1cm}\\ \textbf{if} $(\#(c) = 0)$ \textbf{then}
		\\ \indent$\emptyset$
	\\ \textbf{else} 
		\\\indent  $significado(d,dameUno(c))$ $\cup$ $obtenerSignificado(d,sinUno(c))$
	\\  \textbf{fi} 
\end{description}

\subsection{Algoritmos}
\algoritmo{iNueva}{}{res:redstr}
{
\State $res$ $\larr$ $CrearTupla(compus:conj(compu),conexiones:dicc(compu,dicc(interfaz,compu)))$ \complejidad{O(1)}
\State $res.conexiones$ $\larr$ $Vacio()$  \complejidad{O(1)}
\State $res.compus$ $\larr$ $Vacio()$ \complejidad{O(1)}
\State $return$ $res$
}
{O(1)}

\algoritmo{iAgregarCompu}{in/out r:redstr, in c:compu} {}{
	\State $AgregarRapido(r.compus,c)$ \complejidad{O(1)}
}
{O(1)}

\algoritmo{iConectar}{in/out r:redstr, in c$_0$:compu, in c$_1$:compu, in i$_0$:interfaz, in i$_1$:interfaz}{}{
	\If {$\neg$ $(definido?(r.conexiones,c_0))$}
		\State $definir(r.conexiones,c_0,Vacio())$ \complejidad{O($\# (r.compus)$)}
	\EndIf
	\If {$\neg$ $(definido?(r.conexiones,c_1))$}
		\State $definir(r.conexiones,c_1,Vacio())$ \complejidad{O($\# (r.compus)$)}
	\EndIf
	\State $definir(significado(r.conexiones,c_0),i_0,c_1)$ \complejidad{O($\# (r.compus)$)}
	\State $definir(significado(r.conexiones,c_1),i_1,c_0)$ \complejidad{O($\# (r.compus)$)}
}
{O($\#$ $(r.compus)$)}

\algoritmo{iStringAIndice}{in r:redstr, in c:compu}{res:nat}{
	\State $itCompus$ $\larr$ $CrearIt(r.compus)$ \complejidad{O(1)}
	\\
	\State $//$ $L = longitud de IP mas larga$
	\While {$itCompus.siguiente().IP != c.IP$} \complejidad{O(L)}
		\State $res ++$ \complejidad{O(1)}
		\State $avanzar(itCompus)$ \complejidad{O(1)}
	\EndWhile
	\State $return$ $res$ \complejidad{O(1)}
}
{O($\#(r.compus)*L$)}

\algoritmo{iUsaInterfaz}{in r:redstr, in c:compu, in i:interfaz}{res:bool}{
	\State \If {$definido?(r.conexiones,c)$} \complejidad{O($\#(r.compus)$)}
		\State res $\larr$ $definido?(significado(r.conexiones,c),i)$ \complejidad{O($\#(r.compus)$)}
	\Else
		\State res $\larr$ $false$ \complejidad{O($1$)}
	\EndIf
	\State $return$ $res$ \complejidad{O($1$)}
}
{O($\#(r.compus)$)}

\algoritmo{iConectadas?}{in r:redstr, in c$_1$:compu, in c$_2$:compu}{res:bool}{
	\State $res$ $\larr$ $pertence(c_2,significados(significado(r.conexiones,c_1)))$ \complejidad{O($\#(r.compus)$)}
	\State $return$ $res$ \complejidad{O(1)}
}
{O($\#(r.compus)$)}

\algoritmo{iInterfazUsada}{in r:redstr, in c$_1$:compu, in c$_2$:compu}{res:interfaz}{
	\State $res$ $\larr$ $dameClave(significado(d,c_1),c_2)$ \complejidad{O($\#(r.compus)$)}
	\State $return$ $res$ \complejidad{O(1)}
}
{O($\#(r.compus)$)}

\algoritmo{iCaminoMinimo}{in r:red, in f:compu, in d:compu}{res : Lista(compu)}{
	\Var{n:nat} $\larr$ $Longitud(r.compus)$ \complejidad{O(1)}
	\Var{dist:Arreglo(nat)} $\larr$ $CrearArreglo(n)$ \complejidad{O(n)}
	\Var{prev:Arreglo(nat)} $\larr$ $CrearArreglo(n)$ \complejidad{O(n)}
	\Var{s:nat} $larr$ $stringAIndice(r,f)$ \complejidad{O(L*n)}
	\State $dist[s]$ $\larr$ $0$ \complejidad{O(1)}
	\State $prev[s]$ $\larr$ $NULL$ \complejidad{O(1)}
	\Var{haySiguiente:bool} $\larr$ $\neg(Longitud(r.compus)==0)$ \complejidad{O(1)}
	\Var{it:itConj} $\larr$ $CrearIt(r.compus)$ \complejidad{O(1)}
	\Var{v:nat} $\larr$ $0$ \complejidad{O(1)}
	\State $vd:tupla(nat,nat)$ \complejidad{O(1)}
	\State // $vd_0,vd_1 : Tupla(nat,nat)$
	\State // $vd_0$ $=_\alpha$ $vd_1$ $\iff$ $\pi_1(vd_0) = \pi_1(vd_1)$
	\State // $vd_0$ $<_\alpha$ $vd_1$ $\iff$ $(\pi_2(vd_0)=pi_2(vd_1)$ $\land$ $\pi_1(vd_0) < \pi_1(vd_1))$ $\lor$
	\State // \indent \indent $\ \ \ ((\pi_2(vd_0) < \pi_2(vd_1))$
	\State \Var{cp:cp}$(tupla(nat,nat),<_\alpha)$ \complejidad{O(1)}
	\State // Inicializo las distancias y previos de cada compu \complejidad{O(1)}	
	\While {$haySiguiente$} \complejidad{O($n*log(n)$)}
		\State $haySiguiente$ $\larr$ $haySiguiente?(it)$ \complejidad{O(1)}
		\If {$\neg(v=s))$} \complejidad{O(1)}
			\State $dist[v]$ $\larr$ $infinito()$ \complejidad{O(1)}
			\State $prev[v]$ $\larr$ $NULL$ \complejidad{O(1)}
		\EndIf
		\State $vd$ $\larr$ $CrearTupla(v,dist[v])$ \complejidad{O(1)}
		\State $encolar(cp,vd)$ \complejidad{O(log(n))}
		\State $v++$ \complejidad{O(1)}
		\State $avanzar(it)$ \complejidad{O(1)}
	\EndWhile
	\State // Calculo distancias
	\While {$\neg Vacia?(cp)$} \complejidad{O($L*n^3$)}
		\State \Var{u:nat} $\larr$ $proximo(cp)$ \complejidad{O(1)}
		\State $desencolar(cp)$ \complejidad{O(log(n))}
		\State // Vecinos de $u$ que todavía están en cp
		\Var{vecinos:Lista($compu$)} $\larr$ $nuevosVecinos(r,u,cp)$ \complejidad{O(n)}
		\Var{itVecinos:itLista} $\larr$ $CrearIt(itVecinos)$ \complejidad{O(1)}
		\Var{haySiguiente:bool} $\larr$ $Vacia?(vecinos)$ \complejidad{O(1)}
		\While {haySiguiente} \complejidad{O($L*n^2$)}
			\State $haySiguiente$ $\larr$ $haySiguiente?(itVecinos)$ \complejidad{O(1)}
			\Var{vert:nat} $\larr$ $stringAIndice(anterior(siguiente(itVecinos)))$ \complejidad{O(L*n)}
			\Var{distAux:nat} $\larr$ $dist[\pi_1(u)] + 1$ \complejidad{O(1)}
			\If {distAux $<$ dist[vert]} \complejidad{O(1)}
				\State dist[vert] $\larr$ distAux \complejidad{O(1)}
				\State prev[vert] $\larr$ u \complejidad{O(1)}
			\EndIf
			\State $avanzar(itVecinos)$ \complejidad{O(1)}
		\EndWhile
	\EndWhile
	\State // Agrego desde el destino hacia la fuente las compus, recorriendolas sobre el arreglo prev y agregandolas por ref.
	\State $agAdelante(res,destino)$ \complejidad{O(1)}
	\Var{cactual:nat} $\larr$ $stringAIndice(r,destino)$ \complejidad{O(L*n)}
	\While {$\neg(cactual=s)$} \complejidad{O(n)}
		\State $agAdelante(res,r.compus[prev[cactual]])$ \complejidad{O(1)}
		\State $cactual$ $\larr$ $prev[cactual]$ \complejidad{O(1)}
	\EndWhile
	\If {$stringAIndice(primero(res)) != s$} \complejidad{O(n)}
		\State $res$ $\larr$ $Vacia$
	\EndIf
	\State $return$ $res$
}{O($L*n^3$)}

\algoritmo{iNuevosVecinos}{in r:redstr, in u:tupla, in cp:cp, in dist:Arreglo(nat)}{res : Lista(compu)}{
	\Var{vecinos:Lista(compu)} $\larr$ $significados(r.conexiones,r.compus[\pi_1(u)])$ \complejidad{O(n)}
	\Var{itVecinos:$itLista$} $\larr$ $CrearIt(vecinos)$ \complejidad{O(1)}
	\While {$haySiguiente$} \complejidad{O($L*n^2$)}
		\State $haySiguiente$ $\larr$ $haySiguiente?(itVecinos)$ \complejidad{O(1)}
		\Var{v:nat} $\larr$ $stringAIndice(stringAIndice(vec))$ \complejidad{O(L*n)}
		\Var{vecino:tupla}(nat,nat) $\larr$ $CrearTupla(v,dist[v])$ \complejidad{O(1)}
		\State // \textit{Si el vecino es mayor que el de mas prioridad,}
		\State // \textit{entonces todavia esta en la cola (no fue desencolado)}
		\If {cp.maxP $<$ vecino} \complejidad{O(1)}
			\State AgAdelante(res,r.compus[$\pi_1$(vecino)]) \complejidad{O(1)}
		\EndIf
	\EndWhile
	\State $return$ $ret$ \complejidad{O(1)}
}{O($L*n^2$)}

\subsubsection{Extensión módulo diccionario}

\operacion{significados}{in d:dicc($\alpha_0$ dicc($\alpha_1$ $\alpha_0$)), in c:$\alpha_0$}{res:Lista($\alpha_0$)}
{$definida?(d,c)$}
{$res \igobs significados(d)$}
{Retorna todos los significados de la clave c}
{O($\# claves(significado(d,c))$)}
{Hay aliasing en res}

\algoritmo{significados}{in d:dicc($\alpha_0$ dicc($\alpha_1$ $\alpha_0)$),in c:$\alpha_0$}{res:Lista($\alpha_0$)}{
	\Var{itSignificados:itDicc} $\larr$ $CrearIt(significado(d,c))$ \complejidad{O(1)}
	\Var{haySiguiente:bool} $\larr$ $\# claves(significado(d,c)) != 0$ \complejidad{O(1)}
	\While {haySiguiente} \complejidad{O(n)}
		\State $haySiguiente$ $\larr$ $haySiguiente?(itDicc)$ \complejidad{O(1)}
		\State // Se agregan los elementos por referencia
		\State $AgAtras(res,anteriorSignificado(siguienteSignificado(itDicc)))$ \complejidad{O(1)}
		\State $avanzar(itDicc)$ \complejidad{O(1)}
	\EndWhile
	\State $return$ $res$ \complejidad{O(1)}
}{O(n)}
