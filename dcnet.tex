%\\\\\\\\\\\\!TEX root = tp2.tex

\section{DCNet}

Una DCNet es un sistema que tiene computadoras en red que reciben paquetes que envían a la computadora destino a cada segundo.

\subsection{Interfaz}

\sexc{DCNet}
$\textbf{usa}$  
Compu, Paquete, Red, diccPref, conjLog, conjLogP
\generos{dcnet}

%%%%%%%%%%%%%%%%%%%%%%%%%%%%%%%%%%%%%%%%%%%%%%%%%%%%%%%%%%%%%%%%%%%%%%%%%%%%

\subsubsection*{Operaciones}


\operacion{crearSistema}{in r: red}{res : dcnet}
 {true}
 {$\var{res} \igobs iniciarDCNet(r)$}
 {Crea un sistema DCNet.}
 {O(???)}
 {el campo.red del modulo devuelto es un puntero a la red de entrada}

\operacion{crearPaquete}{in/out s: dcnet,in p: paquete}{}
 {$s \igobs s_0 \land (\forall p_0:paquete,paqueteEnTransito?(p,s))$ ¬$(p_0 \igobs p) \land$ \\ $destino(p) \in compus(red) \land origen(p) \in compus(red) \yluego $ \\ $haycamino?(destino(p),origen(p),red(s))$}
 {$s \igobs crearPaquete(s_0, p)$}
 {Crea un paquete y lo agrega a la computadora correspondiente.}
 {O($L + log(k)$)}
 {}
 
\operacion{avanzarSegundo}{in/out s: dcnet}{}
 {$s \igobs s_0$}
 {$s \igobs avanzarSegundo(s_0)$}
 {Avanza un segundo el sistema. Todas las computadoras envían su respectivo paquete y en consecuencia se actualizar los paquetes en espera de cada una de ellas.}
 {O($n \times\ (L + log(k))$)}
 
\operacion{dameRed}{in s: dcnet}{res :puntero(red)}
 {$true$}
 {$\var{res} \igobs red(s)$}
 {Devuelve la red de DCNet.}
 {O(1)}
 {Devuelve un puntero a la misma red que la que se pas\'o como par\'ametro para crear el sistema}

\operacion{caminoRecorrido}{in s: dcnet,in p: paquete}{res : secu(compu)}
 {$paqueteEnTransito?(s,p)$}
 {$\var{res} \igobs caminoRecorrido(s,p)$}
 {Devuelve el camino recorrido hasta el momento por un paquete.}
 {O($n \times\  log(max(n,k))$)}
 {Devuelve puntero al camino recorrido que se encuentra en CaminosMinimos}

\operacion{cantidadEnviados}{in s: dcnet,in c: compu}{res : nat}
 {$c \in computadoras(red(s))$}
 {$\var{res} \igobs cantidadEnviados(s,c)$}
 {Devuelve la cantidad de paquetes enviados por una computadora.}
 {O($n$)} %es recorrer todas lascompus del vector e ir al heap que tiene guardado cuantos paquetes envio cada una no ?
{}

 
\operacion{enEspera}{in s: dcnet,in c: compu}{res: puntero(conjLogP(paquete)))}
 {$c \in computadoras(red(s))$}
 {$\var{res} \igobs enEspera(s,c)$}
 {Devuelve un iterador a los paquetes de la computadora.}
 {O(L)}
 {Hay aliasing entre res y el conjunto de paquetes de la computadora pasada por parámetro}

 
\operacion{laQueMasEnvio}{in s: dcnet}{res : compu}
 {$true$}
 {$\var{res} \igobs laQueMasEnvio(s,p)$}
 {Devuelve la computadora que más paquetes envió.}
 {O(1)}
 {POR AHORA DEVUELVE EL IP DE LA COMPU}
 

 
\operacion{paqueteEnTransito?}{in s: dcnet, in p:paquete}{res: bool}
{true}
{$\var{res} = paqueteEnTransito?(s,p)$}
{Devuelve true si el paquete se encuentra en transito en el sistema}
{O($n \times log(k)$)}
{}

 
Las complejidades están en función de las siguientes variables:\\
$n$ : la cantidad total de computadoras que hay en el sistema, \\
$L$ : el hostname más largo de todas las computadoras, \\
$k$ : la cola de paquetes más larga de todas las computadoras. 
\\ \\



%%%%%%%%%%%%%%%%%%%%%%%%%%%%%%%%%%%%%%%%%%%%%%%%%%%%%%%%%%%%%%%%%%%%%%%%%%%%%
\subsection{Representación}

\serc{sistema}{
	\\
	\donde{sistema}{\tupla{
			Compus : \mbox{arreglo(\tupla{IP: String , pN: puntero(conjLog(paquete)),pN': puntero(conjLog(paquetePos)), \#Paquetes: nat})}, %esto quedo feo
			CompusPorPref : \mbox{diccPref(compu,\tupla{PorNom: conjLog(paquete), PorPrior: conjLog(paquete),PorNom': conjLog(paquetePos), PorPrior': conjLog(paquetePos)}},
      CaminosMinimos: \mbox{arreglo(arreglo(arreglo(compu)))},
      LaQMasEnvio : \mbox{nat
	},
	Red: \mbox{red}	
	}}
}

esto se puede borrar despues:
aclaracion en compus en cada indice del arreglo esta la compu correspondiente a esa numeracion

\subsubsection*{Invariante de representación}

\begin{enumerate}
	\item Todos los IP de \textit{compus} pertenecen al conjunto de claves de \textit{CompusPorPref} y la longitud de dicho arreglo es igual al cardinal de las claves del diccionario. 
	\item Los pN de las tuplas que tiene el arreglo \textit{compus} apuntan al conjunto de paquetes(PorNom) de un significado en \textit{CompusPorPref} cuya clave es igual al IP de esa posición en el arreglo.
	\item Todos los conjuntos de los significados de \textit{CompusPorPref} son disjuntos dos a dos.
  \item Los conjuntos de los campos de la tupla PorNom, PorPrior son iguales.
  \item La longitud de \textit{CaminosMinimos} es igual a la longitud del arreglo que tiene \textit{CaminosMinimos} en cada posición.
  \item La longitud del arreglo, que tiene un arreglo de \textit{CaminosMinimos} es menor o igual a la longitud de \textit{CaminosMinimos}.
  \item Los elementos del arreglo anteriormente mencionado son IPs del diccionario \textit{CompusPorPref} y no tiene repetidos.
  \item La computadora que más paquetes envió es aquella cuyo índice es igual a \textit{LaQMasEnvio}

\end{enumerate}

\rep{sistema}{s}  \\
 $1.\ \paratodo{String}{s}\ \fun{$def?$}(s, s.CompusPorPref), (\exists c : compu), \fun{$esta?$}(c,s.Compus)\ \land\ \pi_1(c) = s\ \land\ \fun{longitud}(s.Compus) = \#\Fun{claves}(s.CompusPorPref)$\\
 $2.\ \paratodo{compu}{c}\ \fun{esta?}(c, s.Compus), *\pi_2(c) = \fun{obtener}(\pi_1(c),s.CompusPorPref) $\\
 $3.\ \paratodo{String}{s,t}\ \fun{def?}(s, s.CompusPorPref)\ \land\ \fun{def?}(t, s.CompusPorPref)\ \land\ s \neq t \impluego \\ \fun{obtener}(s, s.CompusPorPref)\ \bigcap\ \fun{obtener}(t, s.CompusPorPref) = \emptyset$\\
 $4.\ \paratodo{String}{s}\ \fun{def?}(s, s.CompusPorPref)\ \impluego\ \pi_1(\fun{obtener}(s, s.CompusPorPref)) =\\ \pi_2(\fun{obtener}(s, s.CompusPorPref))$ \\
 $5,6,7.\ (\paratodo{nat}{i,j}),\ 0 \leq i,j < \fun{longitud}(s.CaminosMinimos) \impluego \fun{longitud}(s.CaminosMinimos) = \fun{longitud}(s.CaminosMinimos[i])\ \land\ \fun{longitud}(s.CaminosMinimos[i][j]) < \fun{longitud}(s.CaminosMinimos)\ \land\ (\paratodo{nat}{e}), \fun{esta?}(e, s.CaminosMinimos[i][j]) \Rightarrow \fun{pertenece}(e,s.CompusPorPref)  $\\
 $8. \paratodo{compu}{c}\ \fun{esta?}(c, s.Compus)\ \impluego\ \pi_3(c) \leq \pi_3(s.Compus[s.LaQMasEnvio])$
%\subsubsection*{Función de abstracción}
 %\abs{sistema}{sist}{dcnet}{s} \\
%$(\fun{s.} = \NULL \land \fun{*r.der} = \NULL \ssi nil?(a)) \ \oluego$ \\

% $\fun{*r.izq} = izq(a) \land \fun{*r.der} = der(a) \ \land$
%$r.val = raiz(a) $

%%%%%%%%%%%%%%%%%%%%%%%%%%%%%%%%%%%%%%%%%%%%%%%%%%%%%%%%%%%%%%%%%%%%%%%%%%%%%%%%%%


\subsubsection*{Funci\'on de abstracción}

\abs{dcnet}{DCNet}{s}{dc} \\
$red(dc)=$*$(s.red) \land (\forall c:compu, c\in compus(dc))( enEspera(dc,c)=$*$(enEspera(s,c)) \land$ \\$  cantidadEnviados(dc,c)=cantidadEnviados(s,c)) \land $ \\ $(\forall p:paquete,paqueteEnTransito?(dc,p)) caminoRecorrido(dc,p)=$*$(caminoRecorrido(s,p)) $

\newpage

\subsection{Algoritmos}

\algoritmo{icrearSistema}{in r:red}{res:dcnet}{
%inicializacion Caminos Minimos
	\State $res.red \larr r$
	\State $n \larr \#(\Fun{compus}(red))$
		\complejidad{O($\#$compus(red)=n)?}
	\State $i\larr 0$
	\State $j\larr 0$
		\complejidad{O(1)}
	\State $res.Compus \larr \Fun{CrearArreglo}(n)$
		\complejidad{O(n)}
	\State $res.CaminosMinimos\larr \Fun{CrearArreglo}(n)$
		\complejidad{O(n)}
	\Var{p: arreglo$\_$dimensionable de puntero(conjLog(paquete))}	
	\While{i$<$n}
		\complejidad{O(n)}
		\State $res.CaminosMinimos[i] \larr \Fun{CrearArreglo}(n)$ 
			\complejidad{O(n)}
		\State s:$ \ <nat,conjLog(paquete,<_{id}),conjLog(paquete,<_{p}), $\\$ conjLog(paquetePos,<_{id}),conjLog(paquetePos,<_{p}) >$
		%HAY que ver bien donde definir la relacion!!
		\State $\pi_1(s) \larr compu(r,i)$
		\State $\pi_2(s) \larr \Fun{nuevo}()$
		\State $\pi_3(s) \larr \Fun{nuevo}()$
		\State $\pi_4(s) \larr \Fun{nuevo}()$
		\State $\pi_5(s) \larr \Fun{nuevo}()$
		\State \Fun{definir}(res.CompusPorPref,compu(r,i),s)
			\complejidad{O(L)}
		\State $p[i] \larr pi_3(s)$
		\State $p'[i] \larr pi_5(s) $
		\State $res.Compus[i] \larr $ $<compu(r,i),p[i],p'[i],0>$ 
			\complejidad{O(1)}
		\While{j$<$n}
			\complejidad{O(n)}
			\State $res.CaminosMinimos[i][j] \larr caminoMinimo(compu(r,i),compu(r,j),r)$
				\complejidad{O(complejidad cammin(red))}
			\State $j++$
		\EndWhile
		\State $i++$	
	\EndWhile
	\State $res.LaQMasEnvio \larr 0$
		\complejidad{O(1)}
%holi
}
{O(max\{$n^2 \times O(complejidad cammin(red)),n \times L\}$)}

\algoritmo{icrearPaquete}{in/out s:dcnet,in/out p: paquete}{}{ %hay que cambiar nombres segun heap y avl
  \State $t_1:\ <nat,conjLog(paquete,<_{id}),conjLog(paquete,<_{p}), $\\$ conjLog(paquetePos,<_{id}),conjLog(paquetePos,<_{p}) >$
  \State $t_1 \larr \Fun{Obtener}(origen(p),\ \var{s}.CompusPorPref)$
        \complejidad{O($L$)}
  \State $t_2:\ <nat,conjLog(paquete,<_{id}),conjLog(paquete,<_{p}), $\\$ conjLog(paquetePos,<_{id}),conjLog(paquetePos,<_{p}) >$
  \State $t_2 \larr \Fun{Obtener}(destino(p),\ \var{s}.CompusPorPref)$
  	\complejidad{O($L$)}
  \State $p$'$:paquetePos$ 
  \State $indiceOrigen(p$'$) \larr \pi_1(t_1)$
  	\complejidad{O(1)}
  \State $indiceDestino(p$'$) \larr \pi_1(t_2)$
  	\complejidad{O(1)}
  \State $indiceActual(p$'$) \larr 0$
  \State \Fun{insertar}($\pi_2(t),p$) %agrega segun nombre
        \complejidad{O($log(k)$)}
  \State \Fun{insertar}($\pi_3(t),p$) %agrega segun prioridad
        \complejidad{O($log(k)$)}
  \State \Fun{insertar}($\pi_4(t),p$') %agrega segun nombre
        \complejidad{O($log(k)$)}
  \State \Fun{insertar}($\pi_5(t),p$') %agrega segun prioridad
        \complejidad{O($log(k)$)}
}
{O($L + log(k)$)}


\algoritmo{ilaQueMasEnvio}{in s: dcnet}{res : compu}{
  \State $res \larr s.compus[s.LaQMasEnvio].IP$ 
        \complejidad{O(1)}
  \textbf{ACLARACION: ACA deberia devolver una compu pero es alto bardo y no puedo hacerlo en O(1) me faltan las interfaces}
}
{O(1)}

\algoritmo{idameRed}{in s: dcnet}{res: puntero(red)}{
	\State $res \larr \&(s.red)$
		\complejidad{O(1)}
}
{O(1)}

\algoritmo{ienEspera}{in s:dcnet,in c: compu}{res: puntero(conjLogP(paquete))}{
  \State $t: \ <nat,conjLog(paquete,<_{id}),conjLog(paquete,<_{p}), $\\$ conjLog(paquetePos,<_{id}),conjLog(paquetePos,<_{p}) >$
  \State $t \larr \Fun{Obtener}(\pi_1(c),\ \var{s}.CompusPorPref)$
        \complejidad{O($L$)}
  \State $res \larr \&(\pi_3(t))$
        \complejidad{O(1)}
}
{O($L$)}

\algoritmo{iavanzarSegundo}{in/out s:dcnet}{}{ %FALTA ARREGLAR EL INDICE DE LA QUE MAS ENVIO
  \Var{i: nat} % NO ME COMPILA ESTOOOOO
  \State $i \larr 0$
        \complejidad{O(1)}
  \Var{m: nat}
  \State $m \larr s.Compus[LaQMasEnvio].\#PaqE$
  \While{$i < \Fun{Longitud}(s.Compus)$} %aca supongo que el modulo arreglo tiene la operacion longitud 
        \complejidad{O(n)} 
  	\Var{paqYProxDes: arreglo\_dimensionable de tupla de paquetePos y nat}
  	\State $paqYProxDes \larr \Fun{CrearArreglo}(n)$
    \Var{IP: String}
    \State $IP \larr s.Compus[i].IP$
    \State $t_1: \ <nat,conjLog(paquete,<_{id}),conjLog(paquete,<_{p}), $\\$ conjLog(paquetePos,<_{id}),conjLog(paquetePos,<_{p}) >$
    \State $t_1 \larr \Fun{obtener}(IP, s.CompusPorPref)$
          \complejidad{O(L)}
    \Var{p: paquete}
    \If{$\neg\Fun{vacia?}(\pi_5(t_1))$}%hay paquete para mandar
    \State $p' \larr \Fun{sacarMax}(\pi_5(t_1))$ %tengo en p el paquete que voy a mover
          \complejidad{O($log(k)$)}
    \State \Fun{borrar}($\pi_2(t_1),\ \Fun{paquete}(p')$) %lo borro del avl por prioridad
          \complejidad{O($log(k)$)}
    \State \Fun{borrar}($\pi_3(t_1),\ \Fun{paquete}(p')$) %lo borro del avl por nombre
          \complejidad{O($log(k)$)}
    \State \Fun{borrar}($\pi_4(t_1),\ p' $) %lo borro del avl por prioridad
          \complejidad{O($log(k)$)}
    \State \Fun{borrar}($\pi_5(t_1),\ p'$) %lo borro del avl por nombre
          \complejidad{O($log(k)$)}
    \State $s.Compus[i].\#PaqE \larr s.compus[i].\#PaqE + 1$ 
           \complejidad{O(1)}
    \State $proxima \larr s.CaminosMinimos[\Fun{indiceOrigen}(p')][\Fun{indicedestino}(p')][\Fun{indiceActual}(p')+1]$
    		\complejidad{O(L) (se copia)}
    \textbf{ACLARACION:} coherente con caminoMinimo (sup que da un arreglo de IP)
     \If{$s.Compus[i].\#PaqE > max$}
          \complejidad{O(1)}
      \State $ max \larr i$
           \complejidad{O(1)}
    \EndIf
   \State $paqYProxDes[i] \larr <p',proxima>$
   \Else
   \State $paqYProxDes[i] \larr 0$
   \EndIf
   \State$i \larr i + 1$
  	\complejidad{O(1)}
  \EndWhile
  \State$ s.LaQMasEnvio \larr max$ 
      \complejidad{O(1)}
  \State $i \larr 0$
        \complejidad{O(1)}
  \While{$i < \Fun{Longitud}(s.Compus)$}
  \If{$paqYProxDes[i]\neq 0$}
  	\State $p'\larr \pi_1(paqYProxDes[i])$
  	\complejidad{O(1)}
  	\State $proxima \larr \pi_2(paqYProxDes[i])$
  	\complejidad{O(L)}
    \If{$\neg (\Fun{destino}(\Fun{paquete}(p'))=proxima)$}
    		\complejidad{O(L)}
    \State $\Fun{actualizarIndice}(p')$ %suma 1
    		\complejidad{O(1)}
    \State $t_2: <nat,conjLog(paquete,<_{id}),conjLog(paquete,<_{p}), $\\$ conjLog(paquetePos,<_{id}),conjLog(paquetePos,<_{p}) >$
    \State $t_2 \larr \Fun{obtener}(proxima, s.CompusPorPref)$
             \complejidad{O(L)}
    \State $\Fun{insertar}(\pi_2(t_2),\ \Fun{paquete}(p'))$ 
          \complejidad{O($log(k)$)}
    \State $\Fun{insertar}(\pi_3(t_2),\ \Fun{paquete}(p'))$
          \complejidad{O($log(k)$)}
    \State $\Fun{insertar}(\pi_4(t_2),\ p')$ 
          \complejidad{O($log(k)$)}
    \State $\Fun{insertar}(\pi_5(t_2),\ p')$
          \complejidad{O($log(k)$)}          
    \EndIf
  \EndIf
	\EndWhile

}
{O($ n \times(L + log(k)))$}

\algoritmo{icantidadEnviados}{in/out s:dcnet,in c: compu}{res : nat}{ 
  \Var{i: nat} 
  \State $i \larr 0$
        \complejidad{O(1)}
  \While{$s.compus[i].IP \neq \pi_1(c)$} %este ciclo termina antes de superar la long del array por la pre
        \complejidad{O(n)} 
    \State $i \larr i + 1$
          \complejidad{O(1)}
  \EndWhile
  \State $res \larr s.compus[i].\#PaqE$
        \complejidad{O(1)}
}
{O(n)}

\algoritmo{ipaqueteEnTransito?}{in s: dcnet, in p:paquete}{res: bool}{
	\Var{i: nat} 
  \State $i \larr 0$
        \complejidad{O(1)}
  \Var{b: bool}
  \State $b \larr \neg(\Fun{pertenece?}(p,$*$(s.compus[i].pN)))$
        \complejidad{O($log(k)$)}
        
ESTA BIEN ESTE PERTENECE?
Y CREO que la estructura deberria apuntar a colalog con paquetespos tambien
  \While{$b \land i<n$}
        \complejidad{O(n)} 
    \State $i \larr i + 1$
          \complejidad{O(1)}
    \State $b \larr \neg(\Fun{pertenece?}(p,$*$(s.compus[i].pN)))$
          \complejidad{O($log(k)$)}
  \EndWhile
  \If{$i=n \land b$}
  	\State $res \larr false$
  \Else
  \State $res \larr true$
  \EndIf
}{O($n \times log(k)$}

\algoritmo{icaminoRecorrido}{in s:dcnet,in p: paquete}{res : secu(compu)}{ 
  \Var{i: nat} 
  \State $i \larr 0$
        \complejidad{O(1)}
  \Var{b: bool}
  \State $b \larr \neg(\Fun{pertenece?}(p,$*$(s.compus[i].pN)))$
  ESTA BIEN ESTE PERTENECE?
  Y CREO que la estructura deberria apuntar a colalog con paquetespos tambien
        \complejidad{O($log(k)$)}
  \While{$b$}
        \complejidad{O(n)} 
    \State$i \larr i + 1$
          \complejidad{O(1)}
    \State $b \larr \neg(\Fun{pertenece?}(p,$*$(s.compus[i].pN')))$
          \complejidad{O($log(k)$)}
  \EndWhile
%  \Var{j: nat}
%  \State $j \larr 0$
%        \complejidad{O(1)}
%  \While{ $\neq \pi_3(p)$}
%        \complejidad{O(n)} 
%    \State$j \larr j + 1$
%          \complejidad{O(1)}
%  \EndWhile
  \State $res \larr s.CaminosMinimos[\Fun{indiceOrigen}(p')][i]$
}
{O($n \times log(k)$)}

%%%%%%%%%%%%%%%%%%%%%%%%%%%%%%%%%%%%%%%%%%%%%%%%%%%%%%%%%%%%%%%%%%%%%%%%%%%%%%%%%

\subsection{Servicios Usados}
Del modulo ConjLog requerimos \fun{pertenece}, \fun{buscar}, \fun{sacarMax}, \fun{insertar} y \fun{borrar} en O($log(k)$) .

Del modulo Diccionario Por Prefijos requerimos \fun{Def?}, \fun{obtener} en O($L$).





