%\\\\\\\\\\\\!TEX root = tp2.tex

\section{DCNet}

Una DCNet es

\subsection{Interfaz}

\sexc{DCNet}
$\textbf{usa}$  
Compu, Paquete, Red, diccPref, conjLog, conjLogP
\generos{dcnet}

%%%%%%%%%%%%%%%%%%%%%%%%%%%%%%%%%%%%%%%%%%%%%%%%%%%%%%%%%%%%%%%%%%%%%%%%%%%%

\subsubsection*{Operaciones}


\operacion{crearSistema}{in r: red}{res : dcnet}
 {true}
 {$\var{res} \igobs iniciarDCNet(r)$}
 {Crea un sistema DCNet.}
 {O(?????????????????????????????????????)}

\operacion{crearPaquete}{in/out s: dcnet,in p: paquete}{}
 {$s \igobs s_0 \land FALTA CHOCLO$}
 {$s \igobs crearPaquete(s_0, p)$}
 {Crea un paquete y lo agrega a la computadora correspondiente.}
 {O($L + log(k)$)}

\operacion{avanzarSegundo}{in/out s: dcnet}{}
 {$s \igobs s_0$}
 {$s \igobs avanzarSegundo(s_0)$}
 {Avanza un segundo el sistema. Todas las computadoras envían su respectivo paquete y en consecuencia se actualizar los paquetes en espera de cada una de ellas.}
 {O($n \times\ (L + log(n) + log(k))$)}

\operacion{dameRed}{in s: dcnet}{res : red}
 {$true$}
 {$\var{res} \igobs red(s)$}
 {Devuelve la red de DCNet.}
 {O($????????????????????????????????????$)}

\operacion{caminoRecorrido}{in s: dcnet,in p: paquete}{res : secu(compu)}
 {$paqueteEnTransito?(s,p)$}
 {$\var{res} \igobs caminoRecorrido(s,p)$}
 {Devuelve el camino recorrido hasta el momento por un paquete.}
 {O($n \times\  log(max(n,k))$)}

\operacion{cantidadEnviados}{in s: dcnet,in c: compu}{res : nat}
 {$c \in computadoras(red(s))$}
 {$\var{res} \igobs cantidadEnviados(s,c)$}
 {Devuelve la cantidad de paquetes enviados por una computadora.}
 {O($n$)} %es recorrer todas lascompus del vector e ir al heap que tiene guardado cuantos paquetes envio cada una no ?

\operacion{enEspera}{in s: dcnet,in c: compu}{res: puntero(conjLogP(paquete)))}
 {$c \in computadoras(red(s))$}
 {$\var{res} \igobs enEspera(s,c)$}
 {Devuelve un iterador a los paquetes de la computadora.}
 {O(L)}

\operacion{laQueMasEnvio}{in s: dcnet}{res : compu}
 {$true$}
 {$\var{res} \igobs laQueMasEnvio(s,p)$}
 {Devuelve la computadora que más paquetes envió.}
 {O(1)}

Las complejidades están en función de las siguientes variables:\\
$n$ : la cantidad total de computadoras que hay en el sistema, \\
$L$ : el hostname más largo de todas las computadoras, \\
$k$ : la cola de paquetes más larga de todas las computadoras. 

%%%%%%%%%%%%%%%%%%%%%%%%%%%%%%%%%%%%%%%%%%%%%%%%%%%%%%%%%%%%%%%%%%%%%%%%%%%%%
\subsection{Representación}

\serc{sistema}{
	\\
	\donde{sistema}{\tupla{
			Compus : \mbox{arreglo(\tupla{IP: String , pN: puntero(conjLog(paquete)), \#Paquetes: nat, Num: nat})}, %esto quedo feo
			CompusPorPref : \mbox{diccPref(compu,\tupla{PorNom: conjLog(paquete), PorPrior: conjLog(paquete)}},
      CaminosMinimos: \mbox{arreglo(arreglo(arreglo(compu)))},
      LaQMasEnvio : \mbox{nat
	}}}
}

\subsubsection*{Invariante de representación}

\begin{enumerate}
	\item 
%	\item sino, son operadores
%	\item si el operador es unario, tiene solo hijo izquierdo

\end{enumerate}
%		\rep{restr}{r} \textbf{if} $(r.izq \neq \NULL)$ \textbf{then} \\
%		\textbf{if} ($r.der = \NULL$) \textbf{then} \\
%		$*r.val = "NOT" \land rep(*r.izq)$ \\
%		\textbf{else} $*r.val \in Ag("OR", Ag("AND",vacio)) \land rep(*r.izq) \land rep(*r.der)$
%		\end{enumerate}

%\subsubsection*{Función de abstracción}

 %\abs{sistema}{sist}{dcnet}{s} \\
 %$(\fun{s.} = \NULL \land \fun{*r.der} = \NULL \ssi nil?(a)) \ \oluego$ \\
% $\fun{*r.izq} = izq(a) \land \fun{*r.der} = der(a) \ \land$
%$r.val = raiz(a) $

%%%%%%%%%%%%%%%%%%%%%%%%%%%%%%%%%%%%%%%%%%%%%%%%%%%%%%%%%%%%%%%%%%%%%%%%%%%%%%%%%%

\subsubsection*{Algoritmos}

\algoritmo{icrearPaquete}{in/out s:dcnet,in p: paquete}{}{ %hay que cambiar nombres segun heap y avl
  \State $t: \ < conjLog(paquete),\ conjLog(paquete) >$
  \State $t \larr \Fun{Obtener}(\pi_3(p),\ \var{s}.CompusPorPref)$
        \complejidad{O($L$)}
  \Fun{insertar}($\pi_1(t),p$) %agrega segun nombre
        \complejidad{O($log(k)$)}
  \Fun{insertar}($\pi_2(t),p$) %agrega segun prioridad
        \complejidad{O($log(k)$)}
}
{O($L + log(k)$)}


\algoritmo{ilaQueMasEnvio}{in s: dcnet}{res : compu}{
  \State $res \larr s.LaQMasEnvio$ 
        \complejidad{O(1)}
}
{O(1)}

\algoritmo{ienEspera}{in s:dcnet,in c: compu}{res: puntero(conjLogP(paquete))}{
  \State $t: \ < conjLog(paquete),\ conjLog(paquete) >$
  \State $t \larr \Fun{Obtener}(\pi_1(c),\ \var{s}.CompusPorPref)$
        \complejidad{O($L$)}
  \State $res \larr \&(\pi_2(t))$
        \complejidad{O(1)}
}
{O($L$)}

\algoritmo{iavanzarSegundo}{in/out s:dcnet}{}{ %FALTA ARREGLAR EL INDICE DE LA QUE MAS ENVIO
  \Var{i: nat} % NO ME COMPILA ESTOOOOO
  \State $i \larr 0$
        \complejidad{O(1)}
  \Var{m: nat}
  \State $m \larr s.LaQMasEnvio$
  \While{$i < \Fun{Longitud}(s.compus)$} %aca supongo que el modulo arreglo tiene la operacion longitud 
        \complejidad{O(n)} 
  \If{$\neg\Fun{vacia?}$}
    \Var{IP: String}
    \State $IP \larr \pi_1(s.compu[i])$
    \State $t_1: \ < conjLog(paquete),\ conjLog(paquete) >$
    \State $t_1 \larr \Fun{obtener(IP, s.CompusPorPref)}$
          \complejidad{O(L)}
    \Var{p: paquete}
    \State $p \larr \Fun{sacarMax}(\pi_2(t_1))$ %tengo en p el paquete que voy a mover
          \complejidad{O($log(k)$)}
    $\Fun{borrar}(\pi_2(t_1),\ p)$ %lo borro del avl por prioridad
          \complejidad{O($log(k)$)}
    $\Fun{borrar}(\pi_1(t_1),\ p)$ %lo borro del avl por nombre
          \complejidad{O($log(k)$)}
    $\pi_3(s.compus[i]) \larr \pi_3(s.compus[i]) + 1$
           \complejidad{O(1)}
    \State $t_2: \ < conjLog(paquete),\ conjLog(paquete) >$
    \State $t_2 \larr \Fun{obtener}(\pi_4(P), s.CompusPorPref)$
             \complejidad{O(L)}
    $\Fun{insertar}(\pi_2(t_2),\ p)$ \\
          \complejidad{O($log(k)$)}
    $\Fun{insertar}(\pi_1(t_2),\ p)$
          \complejidad{O($log(k)$)}
    \If{$\pi_3(s.compus[i] > max)$}
          \complejidad{O(1)}
      \State $ max \larr i$
           \complejidad{O(1)}
    \EndIf
  \EndIf
  \State$i \larr i + 1$
  \EndWhile
  \State$ s.LaQMasEnvio \larr max$ 
      \complejidad{O(1)}
}
{O($ n \times(L + log(k)$))}

\algoritmo{icantidadEnviados}{in/out s:dcnet,in c: compu}{res : nat}{ 
  \Var{i: nat} % NO ME COMPILA ESTOOOOO
  \State $i \larr 0$
        \complejidad{O(1)}
  \While{$\pi_1(s.compus[i]) \neq \pi_1(c)$} %este ciclo termina antes de superar la long del array por la pre
        \complejidad{O(n)} 
    \State$i \larr i + 1$
          \complejidad{O(1)}
  \EndWhile
  \State$res \larr \pi_3(s.compus[i])$
        \complejidad{O(1)}
}
{O(n)}

\algoritmo{icaminoRecorrido}{in s:dcnet,in p: paquete}{res : secu(compu)}{ 
  \Var{i: nat} 
  \State $i \larr 0$
        \complejidad{O(1)}
  \Var{b: bool}
  \State $b \larr \neg(\Fun{pertenece?}(p,\pi_3(s.compus[i])))$
        \complejidad{O($log(k)$)}
  \While{$b$}
        \complejidad{O(n)} 
    \State$i \larr i + 1$
          \complejidad{O(1)}
    \State $b \larr \neg(\Fun{pertenece?}(p,\pi_3(s.compus[i])))$
          \complejidad{O($log(k)$)}
  \EndWhile
  \State$res \larr s.CaminosMinimos[\pi_3(p)][\pi_4(s.compus[i])]$
        \complejidad{O(1 o n dependiendo de si hago copia o no)}
}
{O($n \times log(k)$)}

%%%%%%%%%%%%%%%%%%%%%%%%%%%%%%%%%%%%%%%%%%%%%%%%%%%%%%%%%%%%%%%%%%%%%%%%%%%%%%%%%

\subsection{Servicios Usados}


