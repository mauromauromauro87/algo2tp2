\section{ConjLog}

\subsection{Interfaz($\alpha, \igsub{\alpha}, <_\alpha$)}

	\subsubsection{parámetros formales}
	\begin{description}[labelindent=1cm]
	\item[] \textbf{g\'eneros} $\alpha$
	\item[] \textbf{operaciones} 
			\begin{description}[labelindent=1cm] 
			\item[] $\bullet$ $\igsub{\alpha}$ $\bullet$ $:$ $\alpha$ $\times$ $\alpha$ $\rarr$ $bool$ \indent Relación de equivalencia
			\item[] $\bullet$ $<_\alpha$ $\bullet$ $:$ $\alpha$ $\times$ $\alpha$ $\rarr$ $bool$	\indent Relación de orden
			\end{description}
	\end{description}
	
\sexc{Conj($\alpha$)}
\generos{conjLog($\alpha$)}

%%%%%%%%%%%%%%%%%%%%%%%%%%%%%%%%%%%%%%%%%%%%%%%%%%%%%%%%%%%%%%%%%%%%%%%%%%%%

\subsubsection*{Operaciones}

\operacion{nuevo}{}{res:conjLog($\alpha$)}
{true}
{$res \igobs \emptyset$}
{Crea un nuevo conjLog vacio}
{O(1)}
{}

\operacion{vacío?}{in cl:conjLog($\alpha$)}{res:bool}
{true}
{$res = (\emptyset?(cl))$}
{Indica si el conjunto es vacío}
{$O(log(\#(cl)))$}
{}

\operacion{pertenece}{in cl:conjLog($\alpha$),in e:$\alpha$}{res : bool}
{true}
{$res = (e \in cl)$}
{Retorna un booleano que indica si el elemento pertenece al conjunto}
{$O(log(\#(cl)))$}
{}

\operacion{buscar}{in cl:conjLog($\alpha$),in e:$\alpha$}{res : $\alpha$}
{$e \in cl$}
{$res == e$}
{Devuelve el elemento que se está buscando}
{$O(log(\#(cl)))$}
{El elemento se devuelve por referencia y es modificable}

\operacion{menor}{in cl:conjLog($\alpha$), in e:$\alpha$}{res : $\alpha$}
{$e \in cl$}
{$res == max(cl)$}
{Devuelve el menor elemento del conjunto}
{$O(log(\#(cl)))$}
{El elemento se devuelve por referencia y es modificable}

\operacion{insertar}{in/out cl:cl($\alpha$), in e:$\alpha$}{}
{$cl_0 \igobs cl \land \neg(e \in cl)$}
{$cl_0 \igobs Agregar(cl_0, e)$}
{Inserta un nuevo elemento en el conjunto}
{O($log(\#(cl))$)}
{}

\operacion{borrar}{in/out cl:cl($\alpha$), in e:$\alpha$}{}
{$cl_0 \igobs cl \land (e \in cl)$}
{$cl \igobs (cl_0 - \{e\})$}
{Elimina el elemento e del conjunto cl, los iteradores que apunten a este elemento se indefinen}
{O($log(\#(cl))$)}
{}

\subsection{Representación}

\serc{clog}{
	\\
	\donde{clog}{ raiz : \mbox{puntero(nodo)}
	}	
	\donde{nodo}{\tupla{
		der : \mbox{puntero(nodo)},
		izq : \mbox{puntero(nodo)},
		valor : \mbox{$\alpha$},
		padre : \mbox{puntero(nodo)},
		fdb : \mbox{nat}
	}}
}

\newpage
\subsection{Invariante de representación}

\begin{enumerate}
	\item Para todas las raíces, la altura del subárbol derecho menos la altura del subárbol izquierdo de esa raíz es igual al fdb.
	\item El fdb de todas las raíces es 0, 1 o -1.
	\item Si un nodo no es una hoja del árbol entonces los padres de los hijos derecho e izquierdo son iguales y es el nodo
	\item Si un nodo es una hoja del arbol entonces los hijos derecho e izquierdo del árbol son NULL
	\item Para todos los nodos n, todos los nodos del subárbol derecho son mayores que n
	\item Para todos los nodos n, todos los nodos del subárbol izquierdo son menores que n
	\item No hay nodos repetidos
	\item El padre de la raíz es NULL
\end{enumerate}

\subsection{Función de abstracción}

\abs{clog($\alpha$)}{conj($\alpha$)}{cl}{c} \\
$((\paratodo{\alpha}{e}) e \in c \impluego esta(cl,e)) \land size(cl) = \#(c)$

\subsubsection{Aclaración de complejidades}{
	Este módulo es un AVL y fue diseñado para poder cumplir con las especificaciones de complejidad de dcnet.
	Si bien es paramétrico, está pensado para trabajar con tipo 'paquete' en el que la 
	copia se puede realizar en tiempo constante.
	En resumen, todas las operaciones de comparación, asignacion y copia del tipo paramétrico $\alpha$
	deben poder ser realizadas en O(1).
}

\newpage
\subsection{Algoritmos}
\algoritmo{iVacío?}{in cl:conjLog($\alpha$)}{res:bool}{
	\State $res \larr cl == NULL$ \complejidad{O(1)}
}
{O(1)}

\algoritmo{iBorrar}{in/out cl:conjLog($\alpha$), in e : $\alpha$}{}{
	\State \Var{variandoHijoDerecho?:$bool$} $\larr true$
	\State \Var{clactual:$conjLog(\alpha)$} $\larr cl$ \complejidad{O(1)}
	\State \Var{aBorrar:$conjLog(\alpha)$}
	\State \If {$(\neg(cl.der==NULL) \land \neg(cl.izq==NULL))$} \complejidad{O(1)}
		\State $clactual \larr {\Fun{iEncontrarPadre}(clactual, e)}$ \complejidad{O($log(size(cl))$)}
		\If {$cactual.der != NULL \yluego cactual.der.valor==e$}
			\State $aBorrar \larr cactual.der$ \complejidad{O(1)}
		\Else
			\State $aBorrar \larr cactual.izq$ \complejidad{O(1)}
		\EndIf
		\State \Var{mm:$conjLog(\alpha)$} $\larr {\Fun{iDameMayorMenores}(clactual)}$ \complejidad{O($log(size(cl))$)}
		\State \If {$mm.valor == e$} \complejidad{O(1)}
			\State \If{$mm.padre.der!=NULL \yluego mm.padre.der.valor == mm.valor$}
				\State $variandoHijoDerecho? \larr true$ \complejidad{O(1)}
				\State $mm.padre.der = NULL$
				\State $mm.padre.fdb--$
			\Else 
				\State $variandoHijoDerecho? \larr false$ \complejidad{O(1)}
				\State $mm.padre.izq = NULL$
				\State $mm.padre.fdb++$
			\EndIf
		\Else
			\State \Var{mmValor:$\alpha$} $\larr mm.valor$ \complejidad{O(1)}
			\State $aBorrar.valor \larr mmValor$ \complejidad{O(1)}
			\State \If {$mm.izq != NULL$} \complejidad{O(1)}
				\State $mm.valor \larr mm.izq.valor$ \complejidad{O(1)}
				\State $mm.izq \larr NULL$ \complejidad{O(1)}
				\State $mm.fdb++$ \complejidad{O(1)}
				\State \If{$mm.padre.valor == e$}
					\State $variandoHijoDerecho? \larr false$
				\Else
					\State $variandoHijoDerecho? \larr true$
				\EndIf
			\Else
				\State \If{$mm.padre.valor == e$}
					\State $mm.padre.izq = NULL$
					\State $variandoHijoDerecho? \larr false$
				\Else
					\State $mm.padre.der = NULL$
					\State $variandoHijoDerecho? \larr true$
				\EndIf
			\EndIf
		\EndIf
		\State ${\Fun{iRebYRecalcFDB}(mm.padre, variandoHijoDerecho?, estoyBorrando?)}$ \complejidad{O($log(size(cl))$)}
	\Else
		\State \If {$cl.der==NULL \land cl.izq==NULL$} \complejidad{O(1)}
			\State $cl \larr NULL$ \complejidad{O(1)}
		\Else
			\State \If {$cl.der==NULL$} \complejidad{O(1)}
				\State \If {$cl.izq.valor==e$} \complejidad{O(1)}
					\State $cl.izq \larr NULL$ \complejidad{O(1)}
				\Else
					\State $cl.valor \larr cl.izq.valor$ \complejidad{O(1)}
					\State $cl.izq \larr NULL$ \complejidad{O(1)}
				\EndIf
			\Else
				\State \If {$cl.der.valor==e$} \complejidad{O(1)}
					\State $cl.der \larr NULL$ \complejidad{O(1)}
				\Else
					\State $cl.valor \larr cl.der.valor$ \complejidad{O(1)}
					\State $cl.der \larr NULL$ \complejidad{O(1)}
				\EndIf
			\EndIf
		\EndIf
	\EndIf
}
{O($log(size(cl))$)}

\subsubsection*{Justificación de complejidad}{
	Por álgebra de órdenes
}

\algoritmo{iInsertar}{in/out cl:conjLog($\alpha$), in e : $\alpha$}{}{
	\State \Var{clactual:$conjLog(\alpha)$} $\larr cl$ \complejidad{O(1)}
	\State \If {$\neg(cl.der==NULL) \land \neg(cl.izq==NULL)$} \complejidad{$O(1)$}
		\State $clactual \larr {\Fun{iEncontrarPadre}(clactual, e)}$ \complejidad{O($log(size(cl))$)}
		\State \If {$clactual.valor < e$}
			\State $clactual.der \larr $\tupla{der:NULL, izq:NULL, valor:e, padre:clactual, fdb:0} \complejidad{$O(1)$}
			\State ${\Fun{iRebYRecalcFDB}(clactual,true,false)}$
		\Else
			\State $clactual.izq \larr $\tupla{der:NULL, izq:NULL, valor:e, padre:clactual, fdb:0} \complejidad{$O(1)$}
			\State ${\Fun{iRebYRecalcFDB}(clactual,false,false)}$
		\EndIf
	\Else
		\State \If {$cl.der==NULL \land cl.izq==NULL$} \complejidad{O(1)}
			\State $cl \larr $\tupla{der:NULL, izq:NULL, valor:e, padre:clactual, fdb:0} \complejidad{O(1)}
		\Else
			\State \If {$cl.der!=NULL$}
				\State $cl.izq \larr $\tupla{der:NULL,izq:NULL,valor:e,padre:cl,fdb:0} \complejidad{O(1)}
			\Else
				\State $cl.der \larr $\tupla{der:NULL,izq:NULL,valor:e,padre:cl,fdb:0} \complejidad{O(1)}
			\EndIf
		\EndIf
	\EndIf
}
{O($log(size(cl))$)}

\subsubsection*{Justificación de complejidad}{
	Por álgebra de órdenes
}

\algoritmo{iPertenece}{in/out cl:conjLog($\alpha$), in e:$\alpha$}{res:bool}{
	\State \Var{encontrado?:bool} $\larr false$ \complejidad{O(1)}
	\State \Var{clactual:conjLog($\alpha$)} $\larr cl$ \complejidad{O(1)}
	\State \While{$(clactual != NULL) \land \neg(encontrado?)$} \complejidad{O(1)}
		\State \If {$e > clactual.valor$} \complejidad{O(1)}
			\State $clactual \larr clactual.der$ \complejidad{O(1)}
		\Else
			\State \If {$ce < clactual.valor$} \complejidad{O(1)}
				\State $clactual \larr clactual.izq$ \complejidad{O(1)}
			\Else
				\State $encontrado? \larr true$ \complejidad{O(1)}
			\EndIf
		\EndIf
	\EndWhile
	\State $clactual \larr NULL$ \complejidad{O(1)}
	\State $res \larr encontrado?$ \complejidad{O(1)}
}
{$O(log(size(cl)))$}

\subsubsection*{Justificación de complejidad}{
	El ciclo recorre a lo sumo una rama del árbol (el árbol está ordenado), teniendo ésta como máximo 
	la longitud del árbol (sus alturas no difieren en más de una hoja) que es log(n), con n = size(cl)
}

\algoritmo{iMenor}{in cl:conjLog($\alpha$)}{res:$\alpha$}{
	\State \Var{clactual:conjLog(\alpha)} $\larr cl$ \complejidad{O(1)}
	\State $clactual \larr iMenorNodo(clactual)$ \complejidad{O($log(size(cl))$)}
	\State $res \larr clactual.valor$ \complejidad{O(1)}
}
{$O(log(size(cl)))$}

\subsubsection*{Justificación de complejidad}{
	Por álgebra de órdenes
}

\algoritmo{iBuscar}{in cl:conjLog($\alpha$), e:$\alpha$}{res:$\alpha$}{
	\State \Var {padre:conjLog($\alpha$)} $\larr iEncontrarPadre(cl,e)$ \complejidad{O($log(size(cl))$)}
	\State \If {$padre.der != NULL \yluego padre.der.valor==e$} \complejidad{O($1$)}
		\State $res \larr padre.der.valor$
	\Else
		\State $res \larr padre.izq.valor$
	\EndIf
}
{O($log(size(cl))$)}

\subsubsection*{Justificación de complejidad}{
	Por álgebra de órdenes
}

\newpage
\subsection{Auxiliares}
\algoritmo{iRebYRecalcFDB}{in/out cl:conjLog($\alpha$), in variandoHijoDerecho?:bool, in estoyBorrando?:bool}{}{
	\State \Var{clactual:conjLog($\alpha$)} $\larr cl$ \complejidad{$O(1)$}
	\State \Var{termino?:bool} $\larr false$ \complejidad{O(1)}
	\State \If {$estoyBorrando?$}
		\State \While {$clactual != NULL \land \neg(termino?)$} \complejidad {$O(1)$}
			\State \If{$variandoHijoDerecho?$}
				\State \If{$clactual.fdb == -1$} \complejidad{$O(1)$}
					\State \Var{fdbIzq:nat} \complejidad{O(1)}
					\State \If {$cl.izq != NULL$}
						\State $fdbIzq \larr cl.izq$ \complejidad{O(1)}
					\EndIf
					\State \If {$cl.izq != NULL \yluego cl.izq.fdb == 1$} \complejidad{O(1)}
						\State ${\Fun{iRotarLR}(cl.izq)}$ \complejidad{$O(1)$}
					\EndIf
					\State ${\Fun{iRotarLL}(cl)}$ \complejidad{$O(1)$}
					\State \If {$cl.izq!=NULL \yluego fdbIzq==0$} \complejidad{O(1)}
						\State $termino? \larr true$ \complejidad{O(1)}
					\EndIf
				\Else
					\State \If {$cl.fdb == +1$} \complejidad{O(1)}
						\State $cl.fdb \larr 0$ \complejidad{O(1)}
						\State $termino? \larr true$ \complejidad{O(1)}
					\Else
						\State $cl.fdb \larr -1$ \complejidad{$O(1)$}
					\EndIf
				\EndIf
			\Else
				\State \If{$clactual.fdb == -1$} \complejidad{$O(1)$}
					\State \Var{fdbDer:nat} \complejidad{O(1)}
					\State \If{$cl.der != NULL$} \complejidad{O(1)}
						\State $fdbDer \larr cl.der.fdb$ \complejidad{O(1)}
					\EndIf
					\State \If {$cl.der != NULL \yluego fdbDer == 1$} \complejidad{O(1)}
						\State ${\Fun{iRotarRL}(cl.der)}$ \complejidad{$O(1)$}
					\EndIf
					\State ${\Fun{iRotarRR}(cl)}$ \complejidad{$O(1)$}
					\State \If {$fdbDer==0$} \complejidad{O(1)}
						\State $termino? \larr true$ \complejidad{O(1)}
					\EndIf
				\Else
					\State \If {$cl.fdb == -1$} \complejidad{$O(1)$}
						\State $cl.fdb \larr 0$ \complejidad{$O(1)$}
						\State $termino? \larr true$ \complejidad{$O(1)$}
					\Else
						\State $cl.fdb \larr +1$ \complejidad{$O(1)$}
					\EndIf
				\EndIf
			\EndIf
			\State $variandoHijoDerecho \larr (cl.padre != NULL \yluego cl.padre.der.valor == cl.valor)$ \complejidad{O(1)}
			\State $clactual \larr clactual.padre$ \complejidad{$O(1)$}
		\EndWhile
	\Else  $\indent$ // No hubo borrado, entonces hubo una inserción
		\State \While {$clactual != NULL \land \neg(termino?)$} \complejidad {$O(1)$}
			\State \If{$variandoHijoDerecho?$}
				\State \If{$clactual.fdb == +1$} \complejidad{$O(1)$}
					\State \Var{fdbDer:nat} \complejidad{O(1)}
					\State \If {$cl.der != NULL$}
						\State $fdbDer \larr cl.der.fdb$ \complejidad{O(1)}
					\EndIf
					\State \If {$cl.der != NULL \yluego fdbDer == -1$} \complejidad{O(1)}
						\State ${\Fun{iRotarRL}(cl.der)}$ \complejidad{$O(1)$}
					\EndIf
					\State ${\Fun{iRotarRR}(cl)}$ \complejidad{$O(1)$}
					\State $termino? \larr true$ \
				\Else
					\State \If {$clactual.fdb == -1$} \complejidad{$O(1)$}
						\State $clactual.fdb \larr 0$ \complejidad{$O(1)$}
						\State $termino? \larr true$ \complejidad{$O(1)$}
					\Else
						\State $clactual.fdb \larr 1$ \complejidad{$O(1)$}
					\EndIf
				\EndIf
			\Else
				\State \If{$clactual.fdb == -1$} \complejidad{$O(1)$}
					\State $fdbIzq:nat$ \complejidad{$O(1)$}
					\State \If {$cl.izq != NULL$}
						\State $fdbIzq \larr cl.izq.fdb$ \complejidad{$O(1)$}
					\EndIf
					\State \If {$cl.izq != NULL \yluego fdbIzq == +1$} \complejidad{$O(1)$}
						\State ${\Fun{iRotarLR}(cl.izq)}$ \complejidad{$O(1)$}
					\EndIf
					\State ${\Fun{iRotarLL}(cl)}$ \complejidad{$O(1)$}
					\State $termino? \larr true$ \complejidad{$O(1)$}
				\Else
					\State \If {$clactual.fdb == +1$} \complejidad{$O(1)$}
						\State $clactual.fdb \larr 0$ \complejidad{$O(1)$}
						\State $termino? \larr true$ \complejidad{$O(1)$}
					\Else
						\State $clactual.fdb \larr -1$ \complejidad{$O(1)$}
					\EndIf
				\EndIf
			\EndIf
			\State $variandoHijoDerecho \larr (cl.padre != NULL \yluego cl.padre.der.valor == cl.valor)$ \complejidad{O(1)}
			\State $clactual \larr clactual.padre$ \complejidad{$O(1)$}
		\EndWhile
	\EndIf
}
{O($log(size(cl))$)}

\subsubsection*{Justificación de complejidad}{
	En éste ciclo, tanto para el borrado y para la inserción se tiene un nodo interno que fue modificado y a partir de éste se 
	comienza a subir hasta llegar como máximo al nodo raíz del árbol,recorriendo como mucho la altura del árbol
}

\algoritmo{iRotarRR}{in/out cl:conjLog($\alpha$)}{}{
	\State \Var{nietoRR:conjLog($\alpha$)} $\larr cl.der.der$ \complejidad{O(1)}
	\State \Var{hijoDer:conjLog($\alpha$)} $\larr cl.der$ \complejidad{O(1)}
	\State \Var{hijoIzq:conjLog($\alpha$)} $\larr cl.izq$ \complejidad{O(1)}
	\State $cl.der \larr NULL$ \complejidad{O(1)} 
	\State $cl.izq = $\tupla{der:hijoDer.der, izq:cl.izq, valor: cl.valor, padre:cl, fdb:0} \complejidad{O(1)}
	\State $cl.izq.izq.padre \larr cl.izq$ \complejidad{O(1)}
	\State $cl.izq.der.padre \larr cl.izq$ \complejidad{O(1)}
	\State $cl.valor = hijoDer.valor$ \complejidad{O(1)}
	\State $cl.der = nietoRR$ \complejidad{O(1)}
	\State $cl.der.padre \larr cl$ \complejidad{O(1)}
}
{O(1)}

\subsubsection*{Justificación de complejidad}{
	Son operaciones sobre $\alpha$ y punteros
}

\algoritmo{iRotarRL}{in/out cl:conjLog($\alpha$)}{}{
	\State \Var{nietoRR:conjLog($\alpha$)} $\larr cl.der.der$ \complejidad{O(1)}
	\State \Var{nietoRL:conjLog($\alpha$)} $\larr cl.der.izq$ \complejidad{O(1)}
	\State \Var{valorDer:$\alpha$} $\larr cl.der.valor$ \complejidad{O(1)}
	\State $cl.der.valor \larr nietoRL.valor$ \complejidad{O(1)}
	\State $nietoRR.izq \larr nietoRL.der$ \complejidad{O(1)}
	\State $cl.der.der \larr nietoRR$ \complejidad{O(1)}
	\State $cl.der.izq \larr nietoRL.izq$ \complejidad{O(1)}
	\State $cl.der.der.padre \larr cl.der$ \complejidad{O(1)}
	\State $cl.der.izq.padre \larr cl.der$ \complejidad{O(1)}
	\State $cl.izq.fdb \larr +1$
}
{$O(1)$}

\subsubsection*{Justificación de complejidad}{
	Son operaciones sobre $\alpha$ y punteros
}

\algoritmo{iRotarLL}{in/out cl:conjLog($\alpha$)}{}{
	\State \Var{nietoLL:conjLog($\alpha$)} $\larr cl.izq.izq$ \complejidad{O(1)}
	\State \Var{hijoIzq:conjLog($\alpha$)} $\larr cl.izq$ \complejidad{O(1)}
	\State \Var{hijoDer:conjLog($\alpha$)} $\larr cl.der$ \complejidad{O(1)}
	\State $cl.izq \larr NULL$ \complejidad{O(1)} 
	\State $cl.der = $\tupla{der:cl.der, izq:hijoIzq.der, valor: cl.valor, padre:cl, fdb:0} \complejidad{O(1)}
	\State $cl.der.izq.padre \larr cl.der$ \complejidad{O(1)}
	\State $cl.der.der.padre \larr cl.der$ \complejidad{O(1)}
	\State $cl.valor = hijoIzq.valor$ \complejidad{O(1)}
	\State $cl.izq = nietoLL$ \complejidad{O(1)}
	\State $cl.izq.padre \larr cl$ \complejidad{O(1)}
}
{$O(1)$}

\subsubsection*{Justificación de complejidad}{
	Son operaciones sobre $\alpha$ y punteros
}

\algoritmo{iRotarLR}{in/out cl:conjLog($\alpha$)}{}{
	\State \Var{nietoLL:conjLog($\alpha$)} $\larr cl.izq.izq$ \complejidad{O(1)}
	\State \Var{nietoLR:conjLog($\alpha$)} $\larr cl.izq.der$ \complejidad{O(1)}
	\State \Var{valorIzq:$\alpha$} $\larr cl.izq.valor$ \complejidad{O(1)}
	\State $cl.izq.valor \larr nietoRL.valor$ \complejidad{O(1)}
	\State $nietoLL.izq \larr nietoLR.izq$ \complejidad{O(1)}
	\State $cl.izq.izq \larr nietoLL$ \complejidad{O(1)}
	\State $cl.izq.der \larr nietoLR.der$ \complejidad{O(1)}
	\State $cl.izq.izq.padre \larr cl.izq$ \complejidad{O(1)}
	\State $cl.izq.der.padre \larr cl.izq$ \complejidad{O(1)}
	\State $cl.izq.fdb \larr -1$ \complejidad{O(1)}
}
{$O(1)$}

\subsubsection*{Justificación de complejidad}{
	Son operaciones sobre $\alpha$ y punteros
}

\algoritmo{iEncontrarPadre}{in cl:conjLog($\alpha$), e:$\alpha$}{res:conjLog($\alpha$)}{
	\State \Var{clactual:conjLog($\alpha$)} \complejidad{O(1)}
	\State \Var{encontrado?:bool} $\larr (clactual.der != NULL \yluego clactual.der.valor == e) \lor (clactual.izq !=NULL \yluego clactual.izq.valor == e)$
	\State \While {$\neg encontrado?$} 
		\State \If {$e > clactual.valor$}
			\State $clactual \larr clactual.der$ \complejidad{$O(1)$}
		\Else
			\State $clactual \larr clactual.izq$ \complejidad{$O(1)$}
		\EndIf
		\State $encontrado? \larr (clactual.der != NULL \yluego clactual.der.valor == e) \lor (clactual.izq !=NULL \yluego clactual.izq.valor == e)$
	\EndWhile
	\State $res \larr clactual$ \complejidad{$O(1)$}
}
{$O(size(cl))$}

\subsubsection*{Justificación de complejidad}{
	El ciclo recorre a lo sumo una rama del árbol (el árbol está ordenado), teniendo ésta como máximo 
	la altura del árbol (sus alturas no difieren en más de una hoja) que es log(n), con n = size(cl)
}

\algoritmo{iDameMayorMenores}{in cl:conjLog($\alpha$), e:$\alpha$}{res:conjLog($\alpha$)}{
	\State \Var{clactual:conjLog($\alpha$)} $\larr cl$ \complejidad{O(1)}
	\State \If {$clactual.izq != NULL$} \complejidad{O(1)}
		\State $clactual \larr iMayorNodo(clactual)$ \complejidad{$O(log(size(cl)))$}
	\EndIf
	\State $res \larr clactual$ \complejidad{O(1)}
}

\subsubsection*{Justificación de complejidad}{
	Por álgebra de órdenes
}

\algoritmo{iMenorNodo}{in cl:conjLog($\alpha$)}{res:$conjLog(\alpha)$}{
	\State \Var{clactual:conjLog($\alpha$)} $\larr cl$ \complejidad{O(1)}
	\State \While {$clactual.izq != NULL$}
		\State $clactual \larr clactual.izq$
	\EndWhile
	\State $res \larr clactual$ \complejidad{O(1)}
}
{$O(log(size(cl)))$}

\subsubsection*{Justificación de complejidad}{
	Para encontrar el menor nodo se recorre el árbol siempre a la izquierda, se alcanza a recorrer una única rama, 
	es decir la altura del árbol
}

\algoritmo{iMayorNodo}{in cl:conjLog($\alpha$)}{res:$conjLog(\alpha)$}{
	\State \Var{clactual:conjLog($\alpha$)} $\larr cl$ \complejidad{O(1)}
	\State \While {$clactual.der != NULL$}
		\State $clactual \larr clactual.der$
	\EndWhile
	\State $res \larr clactual$ \complejidad{O(1)}
}
{$O(log(size(cl)))$}

\subsubsection*{Justificación de complejidad}{
	Para encontrar el mayor nodo se recorre el árbol siempre a la derecha, se alcanza a recorrer una única rama, 
	es decir la altura del árbol
}

\algoritmo{size}{in cl:conjLog($\alpha$)}{res:$nat$}{
	\State \If {$cl == NULL$}
		\State $res \larr 0$
	\Else 
		\State $res \larr 1 + iSize(cl.der) + iSize(cl.izq)$
	\EndIf
}
{}

\subsection{Operaciones auxiliares de conj($\alpha$)}


\begin{description} 
\item[] $menor:$ $conj(\alpha)$ $c$ $\rarr$ $\alpha$ \setlength{\parindent}{1cm} \indent $\{ \#(c)>0 \}$

\item[] $menor(c)$ $=$ 
	\setlength{\parindent}{1cm}\\ \textbf{if} $(\#(c) = 1)$ \textbf{then} 
		\\ \indent$dameUno(c)$
	\\ \textbf{else} 
		\\\indent  \textbf{if}  $(dameUno(c) < menor(sinUno(c))$ \textbf{then} 
			\\\indent \indent $dameUno(c)$
		\\\indent  \textbf{else} 
			\\\indent \indent$menor(sinUno(c))$
		\\\indent  \textbf{fi} 
	\\  \textbf{fi} 
\end{description}
