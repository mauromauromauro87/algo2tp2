\section{Diccionario por Prefijos}

El m\'odulo Diccionario por prefijos provee un diccionario en el que las claves son secuencias no acotadas de caracteres. Con el se puede definir una clave, obtener un significado y eliminar una clave.
Estas tres operaciones est\'an definidas en tiempo O(L) con L la m\'axima longitud del conjunto de las claves introducidas (y cuando se est\'a definiendo una clave se incluye en el conjunto la clave a introducir).

\subsection{Interfaz}

  \textbf{par\'ametros formales}
  
  \textbf{g\'eneros} $\beta$\\
 


\sexc{Diccionario(secu(char),$\beta$)}
\generos{diccPref(secu(char),$\beta$)}


%%%%%%%%%%%%%%%%%%%%%%%%%%%%%%%%%%%%%%%%%%%%%%%%%%%%%%%%%%%%%%%%%%%%%%%%%%%%

\subsubsection*{Operaciones}

\operacion{nuevo}{}{res:diccPref(secu(char),$\beta$)}
{true}
{($\forall$ p:secu(char)) ¬(def?(p,$res$))}
{Crea un nuevo diccionario vac\'io}
{O(1)}
{}


\operacion{def?}{in dp:diccPref(secu(char){,}$\beta$), in p:secu(char)}{res: bool}
{true}
{res=p $\in$ claves(dp)}
{Devuelve true o false seg\'un si la clave est\'a o no definida}
{O(L)}
{}

\operacion{claves}{in dp:diccPref(secu(char){,}$\beta$)}{res: conj(secu(char))}
{true}
{($\forall$ c: secu(char))c $\in$ claves(dp) $\iff$ def?(c,dp)}
{Devuelve un conjunto de las claves del diccionario}
{O(L)?}
{}

\operacion{definir}{in/out dp:diccPref(secu(char){,}$\beta$), in p:secu(char), in s:$\beta$}{}
{dp=$dp_0$ $\land$ ¬def?(p,dp)}
{def?(p,dp) $\land$ obtener(p,dp)$\igobs$s $\land$ ($\forall$ c $\in$ claves($dp_0$)) def?(c,dp)}
{Inserta una nueva clave con su significado en el diccionario}
{O(L)}
{}


\operacion{obtener}{in dp:diccPref(secu(char){,}$\beta$),in p:secu(char)}{res:$\beta$}
{def?(p,dp)}
{$res = obtener(p,dp)$}
{Retorna el significado de la clave pedida}
{O(L)}
{Devuelve res por referencia}


\operacion{eliminar}{in/out dp:diccPref(secu(char){,}$\beta$),in p:secu(char)}{}
{dp=$dp_0$ $\land$ def?(p,dp)}
{¬def?(dp) $\land (\forall c \in claves(dp_0),c\neq$ p)def?(c,dp)}
{Elimina del diccionario la clave deseada}
{O(L)}
{}
