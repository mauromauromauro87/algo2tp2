\section{PaquetePos}

Un PaquetePos es

\subsection{Interfaz}

\sexc{\tupla{:Paquete,:nat,:nat,:nat}}
\generos{paquetePos}

%%%%%%%%%%%%%%%%%%%%%%%%%%%%%%%%%%%%%%%%%%%%%%%%%%%%%%%%%%%%%%%%%%%%%%%%%%%%

\subsubsection*{Operaciones}

\operacion{crearPaquete}{in id: nat,in o: compu,in d: compu,in pr: nat}{res:$<paquete,nat,nat,nat>$}
 {true}
 {$pi_1(pi_1(res))=id \land pi_2(pi_1(res))=pr \land pi_3(pi_1(res))=o \land pi_4(pi_1(res))=d$}
 {Crea un paquete}
 {O(1)}
 {}
 
 \operacion{\argumento $<_{p}$ \argumento}{in $p_1$:${<paquete,nat,nat,nat>}$,in $p_2$:${<paquete,nat,nat,nat>}$}{res: bool}
 {true}
 {res=true $\iff$ $(\pi_2(\pi_1(p_1)) = \pi_2(\pi_1(p_2)) \land \pi_1(\pi_1((p_1)) < \pi_1(\pi_1((p_2)) \lor (\pi_1(\pi_1((p_1)) < \pi_1(\pi_1((p_2))))$}
 {Define un orden en paquete seg\'un la prioridad}
 {O(1)}
 {}
 
 \operacion{\argumento $<_{id}$ \argumento}{in p$_1$:${<paquete,nat,nat,nat>}$, in p$_2$:${<paquete,nat,nat,nat>}$}{res: bool}
 {true}
 {res=true $\iff$ ($pi_1(pi_1(p_1))<(pi_1(pi_1(p_2)))$)}
 {Define un orden en paquete seg\'un el id}
 {O(1)}
 {}
 
 \operacion{getPaquete}{in ppos:${<paquete,nat,nat,nat>}$}{res:paquete}
 {true}
 {$res \igobs \pi_1(paquete)$}
 {Getter de paquete}
 {O(1)}
 {}
 
 \operacion{indiceOrigen}{in p:${<paquete,nat,nat,nat>}$}{res:nat}
 {true}
 {$res = \pi_2(p)$}
 {Getter de indiceOrigen}
 {O(1)}
 {}
 
 \operacion{indiceDestino}{in p:${<paquete,nat,nat,nat>}$}{res:nat}
 {true}
 {$res = \pi_3(p)$}
 {Getter de indiceDestino}
 {O(1)}
 {}  
 
 \operacion{posActual}{in p:${<paquete,nat,nat,nat>}$}{res:nat}
 {true}
 {$res = \pi_4(p)$}
 {Getter de posActual}
 {O(1)}
 {} 
  
 \operacion{actualizarPosActual}{in/out p:${<paquete,nat,nat,nat>}$}{} 
 {$p_1 \igobs p$}
 {$posActual(p) = posActual(p_1) + 1$}
 {Aumentar la posicion actual}
 {O(1)}
 {}
  
\subsection{Representación}

\serc{paqPos}{
\\
\donde{paqPos}{\tupla{
	paquete: \mbox{paquete},
	indiceOrigen: \mbox{nat},
	indiceDestino: \mbox{nat},
	posActual: \mbox{nat}
}}}

\subsubsection*{Justificación de estructura}{
	Las restricciones de complejidad del tipo dcnet para caminoRecorrido y enEspera nos obligaron a tener representadas dos estructuras
	distintas para almacenar los paquetes. En una de ellas tenemos los paquetes a retornar por la operación enEspera que devuelve los 
	paquetes en la cola de cierta computadora y en la otra devolvemos una estructura similar a paquete (paquetePos) con información
	adicional acerca de las posiciones en que se encuentra el paquete que nos permite encontrar el caminoRecorrido en el tiempo solicitado.
}
